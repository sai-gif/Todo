\documentclass{article}
\usepackage[a4paper,margin=1in]{geometry}
\usepackage{hyperref}
\usepackage{titlesec}
\usepackage{enumitem}
\usepackage{graphicx}


\title{\textbf{React Native Development Tasks}}
\author{Sai Kumar Reddy Kaluvakolu}
\date{ID: 001258100 \\ Repo Link: \href{https://github.com/sai-gif/Todo}{https://github.com/sai-gif/Todo}}

\begin{document}

\maketitle

\section*{Task 1}

\subsection*{Q1.1 : Attach screenshots of your app running on an emulator and on a physical Android
 or iOS device.}

 \textbf{A:}
    \begin{itemize}
        \item All the images have been uploaded at the end of the document
    \end{itemize}





\subsection*{Q1.2: Describe any differences you observed between running the app on an emulator versus a physical device.}
\textbf{A:}
\begin{itemize}
    \item \textbf{Emulator:}
    \begin{itemize}
        \item Setup takes a lot of time and requires significant computational resources.
        \item Allows testing multiple devices.
    \end{itemize}
    \item \textbf{Physical Device:}
    \begin{itemize}
        \item Real-world performance testing.
        \item Limited to one device at a time.
    \end{itemize}
\end{itemize}

\subsection*{Q2.1: Explain the steps you followed to set up an emulator in Android Studio or Xcode.}
\textbf{A:}
\begin{enumerate}
    \item Installed Java and set the system path.
    \item Installed android studio and setup sdk 
    \item Used Android Studio’s AVD Manager to create an emulator.
    \item Created a new React Native project:
    \begin{itemize}
        \item Attempted \texttt{npm install -g react-native-cli} (deprecated and did not work).
        \item Used updated steps from the React Native documentation:
        \begin{verbatim}
        npx @react-native-community/cli@latest init todoapp
        \end{verbatim}
    \end{itemize}
    \item Configured Android environment variables such as \texttt{ANDROID\_HOME} along with other env involved.
    \item Resolved Gradle issues with the following commands:
    \begin{verbatim}
    ./gradlew clean
    ./gradlew assembleDebug --info
    npx react-native start --reset-cache
    \end{verbatim}
    \item Ran the application with npm run android
\end{enumerate}

\subsection*{Q2.2: Discuss any challenges you faced during the setup and how you overcame them.}
\textbf{A:}
\begin{itemize}
    \item \textbf{Challenges:}
    \begin{itemize}
        \item \texttt{react-native-cli} was deprecated, and the professor's commands did not work.
        \item Missing Android environment variables caused \texttt{npm run android} to fail.
        \item The emulator did not launch automatically and threw errors.
    \end{itemize}
    \item \textbf{Resolutions:}
    \begin{itemize}
        \item Used updated CLI commands from React Native documentation.
        \item Set environment variables (\texttt{ANDROID\_HOME}, \texttt{PATH}) using tutorials.
        \item Manually launched the emulator and cleared Gradle caches.
    \end{itemize}
\end{itemize}

\subsection*{Q3.1: Describe how you connected your physical device to run the app using Expo.}
\textbf{A:}
\begin{itemize}
    \item Installed the Expo app from the Play Store.
    \item ran react native application using expo Framework 
    \item Scanned the QR code provided in the terminal after running the app.
    \item Followed the Expo tutorial for a seamless setup without issues: 
    \item https://docs.expo.dev/tutorial/create-your-first-app/
    \href{https://docs.expo.dev/tutorial/create-your-first-app/}{click}.
\end{itemize}

\subsection*{Q3.2: Include any troubleshooting steps if you encountered issues. }
\textbf{A:}
\begin{itemize}
    \item Fortunately, everything was smooth when setting up the app on Expo.
\end{itemize}

\subsection*{Q4.1 and 4.2: Compare and contrast using an emulator versus a physical device for React Native development.}
\textbf{A:}
\begin{itemize}
    \item Set up takes a lot of time on an emulator and requires significant computational resources. Using an emulator, we can test multiple devices, whereas physical devices are limited to one.
    \item \textbf{Emulator:}
    \begin{itemize}
        \item \textbf{Advantages:} Cost-effective, supports multiple devices, built-in debugging tools.
        \item \textbf{Disadvantages:} Slower performance, limited hardware testing, less accurate representation.
    \end{itemize}
    \item \textbf{Physical Device:}
    \begin{itemize}
        \item \textbf{Advantages:} Real-world performance testing, accurate hardware, and sensor usage.
        \item \textbf{Disadvantages:} Requires physical devices, less convenient for testing multiple OS versions.
    \end{itemize}
\end{itemize}


\subsection*{Q 5: Identify a common error you encountered when starting your React Native app. Explain the cause of the error and the steps you took to resolve it.}
\textbf{A:}
\begin{itemize}
    \item \textbf{Error Encountered:}
    \begin{itemize}
        \item \texttt{npm install -g react-native-cli} is deprecated, and the commands provided by the professor did not work.
        \item Running \texttt{npm run android} failed due to missing Android environment variables.
        \item The emulator failed to launch and caused build errors.
    \end{itemize}

    \item \textbf{Cause of the Errors:}
    \begin{itemize}
        \item \textbf{Deprecated CLI Installation:} The \texttt{react-native-cli} is no longer supported, and the latest community CLI must be used.
        \item \textbf{Missing Environment Variables:} The Android development environment was not properly configured (e.g., \texttt{ANDROID\_HOME}, \texttt{PATH} variables missing).
        \item \textbf{Gradle Build Issues:} Stale caches and leftover build artifacts caused build failures.
        \item \textbf{Emulator Launch Issues:} The emulator did not launch automatically and required manual intervention.
    \end{itemize}

    \item \textbf{Steps Taken to Resolve the Errors:}
    \begin{enumerate}
        \item Used the updated CLI commands from React Native documentation:
        \begin{verbatim}
        npx @react-native-community/cli@latest init todoapp
        \end{verbatim}
        \item Configured environment variables:
        \begin{verbatim}
        export ANDROID_HOME=/path/to/Android/Sdk
        export PATH=$PATH:$ANDROID_HOME/emulator
        export PATH=$PATH:$ANDROID_HOME/platform-tools
        \end{verbatim}
        \item Cleaned and rebuilt the project using Gradle:
        \begin{verbatim}
        ./gradlew clean
        ./gradlew assembleDebug --info
        \end{verbatim}
        \item Manually started the emulator using Android Studio or terminal commands:
        \begin{verbatim}
        emulator -avd YourEmulatorName
        \end{verbatim}
        \item Reset the Metro bundler cache to ensure a clean start:
        \begin{verbatim}
        npx react-native start --reset-cache
        \end{verbatim}
    \end{enumerate}
\end{itemize}


\section*{Task 2}

\subsection*{Q2.1: How did you implement marking tasks as complete?}
\textbf{A:}
\begin{itemize}
    \item Modified task objects to include a \texttt{completed} property:
    \begin{verbatim}
    { id: Date.now().toString(), text: task, completed: false }
    \end{verbatim}
    \item Created a custom checkbox component.
    \item Styled completed tasks with strikethrough and grey text using:
    \begin{verbatim}
    textDecorationLine: 'line-through', color: 'grey'
    \end{verbatim}
\end{itemize}

\subsection*{Q2.2: How did you implement data persistence using AsyncStorage?}
\textbf{A:}
\begin{itemize}
    \item Used \texttt{AsyncStorage} to save tasks whenever the state changed. This was achieved using the \texttt{useEffect} hook to monitor changes in the \texttt{tasks} state:
    \begin{verbatim}
    useEffect(() => {
        saveTasks(tasks);
    }, [tasks]);
    \end{verbatim}
    \item The \texttt{saveTasks} function serialized the tasks to JSON and stored them using \texttt{AsyncStorage.setItem}:
    \begin{verbatim}
    try {
        const jsonValue = JSON.stringify(tasks);
        await AsyncStorage.setItem('tasks', jsonValue);
    } catch (e) {
        console.error("Failed to save tasks.");
    }
    \end{verbatim}
    \item Loaded tasks when the page was first rendered by retrieving them with \texttt{AsyncStorage.getItem}:
    \begin{verbatim}
    const loadTasks = async () => {
        try {
            const jsonValue = await AsyncStorage.getItem('tasks');
            setTasks(jsonValue != null ? JSON.parse(jsonValue) : []);
        } catch (e) {
            console.error("Failed to load tasks.");
        }
    };
    \end{verbatim}
    \item Addressed an edge case where no tasks were stored in \texttt{AsyncStorage}. This was handled by checking for \texttt{null} and returning an empty array:
    \begin{verbatim}
    const loadTasks = async () => {
        try {
            const jsonValue = await AsyncStorage.getItem('tasks');
            setTasks(jsonValue != null ? JSON.parse(jsonValue) : []);
        } catch (e) {
            console.error("Failed to load tasks.");
        }
    };
    \end{verbatim}
\end{itemize}


\subsection*{Q2.3: How did you implement task editing?}
\textbf{A:}
\begin{itemize}
    \item Allowed users to edit a task's content by tapping on it. Only one task can be edited at a time, ensuring clarity and focus during the editing process.

    \item Managed the UI for editing tasks by conditionally rendering a \texttt{TextInput} in place of the task text when a task is in editing mode. The specific task is identified using its \texttt{id}, and the \texttt{editingTaskId} and \texttt{editingText} states are updated accordingly.

    \item Rendered the editing interface using the following component structure:
    \begin{verbatim}
    <View style={styles.editContainer}>
        <TextInput
            style={styles.input}
            value={editingText}
            onChangeText={setEditingText}
            onBlur={cancelEditTask}  % Cancels editing if clicked outside
        />
        <TouchableOpacity onPress={() => confirmEditTask(item.id)}>
            <Text style={styles.okButton}>OK</Text>  % Confirms changes
        </TouchableOpacity>
    </View>
    \end{verbatim}

    \item Implemented an update function to modify the task in the state array. The function updates the specific task and clears the editing mode:
    \begin{verbatim}
    const confirmEditTask = (id) => {
        setTasks(tasks.map(task =>
            task.id === id ? { ...task, text: editingText } : task
        ));
        setEditingTaskId(null);
        setEditingText('');
    };
    \end{verbatim}

    \item Handled two primary actions:
    \begin{itemize}
        \item \textbf{Select:} When a task is tapped, it enters editing mode, and the \texttt{TextInput} is displayed with the task's current text pre-filled.
        \item \textbf{Unselect:} Clicking outside the \texttt{TextInput} triggers the \texttt{onBlur} event, which discards any changes and exits editing mode.
    \end{itemize}

    \item Key interactions:
    \begin{itemize}
        \item \textbf{OK Button:} Explicitly saves the updated task content and exits editing mode.
        \item \textbf{Click Outside (Blur):} Cancels the editing and resets the \texttt{editingTaskId} and \texttt{editingText} states.
    \end{itemize}

    \item This approach ensures that task edits are controlled, changes are only saved when explicitly confirmed by the user, and accidental modifications are prevented.
\end{itemize}


\subsection*{Q2.4: Describe the animations you implemented.}
\textbf{A:}
\begin{itemize}
    \item Used the Animated API to create fade-out effects when deleting tasks:
    \begin{verbatim}
    const fadeOut = new Animated.Value(1);
    Animated.timing(fadeOut, {
        toValue: 0,
        duration: 300,
        useNativeDriver: true,
    }).start(() => {
        // Remove task from the list
    });
    \end{verbatim}
    \item \textbf{Key Benefits:}
    \begin{itemize}
        \item Smooth transitions.
        \item Enhanced user experience.
        \item Reduced cognitive load.
    \end{itemize}
\end{itemize}

\section*{Disclosure}
During the setup and development of the React Native app, I utilized AI tools to assist in troubleshooting and resolving errors. These tools were instrumental in identifying updated commands, configuring environment variables, and addressing issues such as:
\begin{itemize}
    \item Resolving deprecated commands for setting up the React Native environment.
    \item Configuring Android emulator environment variables.
    \item Debugging Gradle build errors and emulator launch issues.
    \item Providing guidance on styling components, such as creating a custom checkbox.
\end{itemize}
The AI support helped streamline the debugging process and ensured a smoother development experience.



\begin{figure}[ht!]
    \centering
    \includegraphics[width=0.5\textwidth, height=0.3\textheight]{figs/emulator.jpg} % Adjust width and height as needed
    \caption{App running on an Android emulator.}
    \label{fig:emulator-screenshot}
\end{figure}



\begin{figure}[ht!]
    \centering
    \includegraphics[width=0.5\textwidth, height=0.5\textheight]{figs/mobiletodo.jpg} % Adjust width and height as needed
    \caption{App running on an Android physical device.}
    \label{fig:mobile-screenshot}
\end{figure}

\begin{figure}[ht!]
    \centering
    \includegraphics[width=0.5\textwidth, height=0.5\textheight]{figs/add.jpg} % 
    \caption{Add tasks in the app}
    \label{fig:mobile-screenshot-add}
\end{figure}

\begin{figure}[ht!]
    \centering
    \includegraphics[width=0.5\textwidth, height=0.5\textheight]{figs/1.jpg} % 
    \caption{completing tasks in the app}
    \label{fig:mobile-screenshot-complete}
\end{figure}

\begin{figure}[ht!]
    \centering
    \includegraphics[width=0.5\textwidth, height=0.5\textheight]{figs/2.jpg} % 
    \caption{remove tasks in the app}
    \label{fig:mobile-screenshot-remove}
\end{figure}

\end{document}